\documentclass{article}
\usepackage{authblk}
\usepackage{acro}
\usepackage{amsmath}
\usepackage{amsfonts}

\title{
  A graph-based practice of evaluating collective idendities of cell clusters
}
\author[1,2]{Yuji Okano}
\affil[1]{
  Department of Extended Intelligence for Medicine, 
  The Ishii-Ishibashi Laboratory, 
  Keio University School of Medicine
}
\affil[2]{
  Division of CNS Regeneration and Drug Discovery,
  International Center for Brain Science (ICBS), 
  Fujita Health University
}
\date{\today}

\DeclareAcronym{scRNA-seq}{short = scRNA-seq, long  = single-cell RNA-sequencing}
\DeclareAcronym{GRN}{short = GRN, long  = gene regulatory network}
\DeclareAcronym{JIM}{short = JIM, long  = Jaccard index matrix}

\begin{document}

\maketitle

\section*{Abstract}
Random sentences

\section*{Introduction}
It has been more than 10 years since the birth of \ac{scRNA-seq},
and the technology now is recognized as a prominent game changer
of modern molecular biology. Likewise the pioneering technology, bulk RNA-seq,
scRNA-seq can observe multidimensional gene expression profiles, while
it also can provide such information in single-cell-level.

$\vdots$

In our previous research, we proposed a \ac{GRN}-based
representation of cell clusters while edges of GRNs explain statistical
dependencies between two genes\cite{okano2023set}.

\section*{Results}
Quite interesting, while being misterious.

\section*{Discussion}
I have no idea.

\section*{Methods}
\subsection*{GO term-assisted gene selection}
Although we intended to require experimenters to curate marker genes to use in GRN

\subsubsection*{Jaccard Index}
\begin{equation}\label{jaccard}
  J(A, B) := \frac{A\cap B}{A\cup B}
\end{equation}
Jaccard Index of two sets $A, B$ is defined as Eq. \eqref{jaccard}. We expanded this 
definition to pairwise comparisons of multiple elements by forming a matrix 
where each element is the corresponding Jaccard Index, and we named the matrix \ac{JIM}. For example, the element in 
$i$-th row and $j$-th column (where $i, j, k\in\mathbb{N}$ and $i\leq k, j\leq k$), $JIM_{i,j}$, can be defined as
follows when a JIM of sets $X_1, X_2,\cdots, X_k$ are considered:
\begin{equation}\label{jim}
  JIM_{i, j} := J(X_i, X_j)
\end{equation}
Especially for seed markers, sets of subscribed GO terms (let $G_1, \cdots, G_k$)
and their JIM are calulated in order to set $min_{i,j}(J(G_i, G_j))$ as a threshold of 
biological correspondence.

\subsubsection*{Impletemtation}
Querying the GO database by using mygene.py\cite{mygene}, we calculated the JIM of the
related GO terms of given seed markers with Numpy\cite{numpy}.

\subsection*{scRNA-seq data analysis}
The scRNA-seq data we used in this research were publicly available as online
resources.
\subsubsection*{Preprocessing, Dimensionality Reduction, and Visualization}
We performed $\cdots$ using Scanpy\cite{scanpy}, Polars.

\section*{Abbreviations}
\printacronyms[heading=Abbreviations]

\bibliographystyle{ieeetr}
\bibliography{refs.bib}
\end{document}
